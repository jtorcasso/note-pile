\documentclass{article}
\usepackage{listings}
\usepackage{xcolor}
\usepackage{hyperref}
\usepackage{enumitem}
\hypersetup{linkcolor=blue,colorlinks=true,citecolor=blue,urlcolor=blue}

\newcommand{\inlineterminal}[1]{\colorbox{black}{\lstinline[basicstyle=\ttfamily\color{white}]|#1|}}

\begin{document}
    \begin{enumerate}[itemsep=10pt,parsep=2pt]
        \item Download vagrant
        \item Download virtualbox
        \item Download the box at \url{http://peisenha.blob.core.windows.net/share/cbAnalysis-recomputation.box}
        \item Open terminal and cd into the directory of the downloaded box. If downloaded
        into Downloads: \hfill \\ 

        \inlineterminal{\$ cd Downloads}
        \item Add the box \hfill \\

        \inlineterminal{\$ vagrant box add recompute cbAnalysis-recomputation.box}
        \item Initialize the vagrant environment \hfill \\

        \inlineterminal{\$ vagrant init recompute}
        \item Start the virtual machine \hfill \\

        \inlineterminal{\$ vagrant up}
        \item Connect to the virtual machine \hfill \\
        
        \inlineterminal{\$ vagrant ssh}
        \item Now you are ``inside'' the virtual machine, at the terminal interface. From
        here you can investigate the code used to generate the results. To recompute
        results, we added a command. \hfill \\
        
        \inlineterminal{\$ cbAnalysis-recompute} \hfill \\

        The computation will take a few minutes.
        \item Check the results once computation is complete by checking
        the \inlineterminal{rslt} folder.
        \item Clean-up the results \hfill \\
        
        \inlineterminal{\$ cbAnalysis-clean}

        \item Exit the virtual machine \hfill \\

        \inlineterminal{\$ exit} \hfill \\

        Now you are back to your machine.

        \item You can destroy all traces of the virtual machine. \hfill \\

        \inlineterminal{\$ vagrant destroy}
    \end{enumerate}
\end{document}