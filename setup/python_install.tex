\documentclass{article}
\usepackage{tikz}
\usetikzlibrary{shapes,backgrounds,fit,decorations.pathreplacing}
\usepackage{graphicx}
\usepackage{geometry}
\usepackage{lmodern}
\usepackage{hyperref}
\hypersetup{colorlinks=true,
			urlcolor=blue,
			citecolor=blue,
			linkcolor=blue}
\usepackage{natbib}
\usepackage{listings}
\usepackage{subfig}

\newcommand{\inlinecode}[1]{%
\colorbox{black}{\lstinline[basicstyle=\ttfamily\color{white}]|#1|}}

\begin{document}
	\section*{Installing Packages}
	First, some general comments before installation. 

	How does Python know where to look for Python modules? First it
	looks in current directory (where the script was run), then it looks
	in the Python standard library, then it looks in the directories
	listed in your PYTHONPATH environment variable (which you may
	have not set) and finally in those directories listed in your
	PATH environment variable. So there's no magic involved. One
	thing you have to be observant of is editors which may alter
	these paths before execution. 

	In general, remember that (usually) by default Python packages
	are installed in directories that require administrator
	privileges, which is why we prefix most of the install commands
	with \inlinecode{sudo}. For all of the install commands below
	you can provide additional command arguments to specify an
	alternative install directory to the default.

	Recall three methods to install Python packages:

	\begin{enumerate}
		\item Install using Pip. 

		\inlinecode{\$ sudo pip install [package]}

		And to upgrade the package.

		\inlinecode{\$ sudo pip install --upgrade [package]}

		\item Install using easy\_install

		\inlinecode{\$ easy\_install [package]}

		And to upgrade.

		\inlinecode{\$ easy\_install --upgrade [package]}

		\item Installing from source. Every package comes with a 
		script you can run yourself for installation. So you can
		download compressed folders containing all of the source
		code and then run the installation script yourself. 

		All you need to do to install from source is download
		the files (uncompress if you need to) then run the setup
		script included in these files. 

		\inlinecode{\$ sudo setup.py install}
		
		The package will be installed into the directory where
		your other packages reside. 

		Sometimes
		installation requires a C or Fortran compiler because there
		is source code written in these languages which needs to be
		compiled before you can use it. Numpy and Scipy
		depend on optimized C and Fortran libraries that require
		compilation. Pre-compiled binaries are sometimes available
		online to make installation easy. I think they are usually
		packaged as executables. But hopefully in the next step
		you learn how to successfully install C and Fortran compilers
		on your mac.


		\section*{Need C and Fortran Compilers?}

		On a mac, it appears you need to install Xcode in order
		to install these compilers. Here is one 
		\href{http://www.cyberciti.biz/faq/howto-apple-mac-os-x-install-gcc-compiler/}{link} 
		on how to install gcc (the C compiler). It appears you should
		be able to install gfortran (the Fortran compiler) in a similar
		way, but this 
		\href{http://gcc.gnu.org/wiki/GFortranBinariesMacOS}{link} 
		provides resources on further details.


	\end{enumerate}
\end{document}